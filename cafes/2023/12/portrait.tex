%%%%%%%%%%%%%%%%%%%%%%%%%%%%%%%%%%%%%%%%%
% Jacobs Portrait Poster
% LaTeX Template
% Version 1.0 (31/08/2015)
% (Based on Version 1.0 (29/03/13) of the landscape template
%
% Created by:
% Computational Physics and Biophysics Group, Jacobs University
% https://teamwork.jacobs-university.de:8443/confluence/display/CoPandBiG/LaTeX+Poster
% 
% Further modified by:
% Nathaniel Johnston (nathaniel@njohnston.ca)
%
% Portrait version by:
% John Hammersley
%
% The landscape version of this template was downloaded from:
% http://www.LaTeXTemplates.com
%
% License:
% CC BY-NC-SA 3.0 (http://creativecommons.org/licenses/by-nc-sa/3.0/)
%
%%%%%%%%%%%%%%%%%%%%%%%%%%%%%%%%%%%%%%%%%

%----------------------------------------------------------------------------------------
%	PACKAGES AND OTHER DOCUMENT CONFIGURATIONS
%----------------------------------------------------------------------------------------

\documentclass[final]{beamer}

%% Changer la valeur du scale si le texte est trop grand ou trop petit
\usepackage[scale=2.4]{beamerposter} % Use the beamerposter package for laying out the poster
\usepackage{pifont}
\usepackage{wrapfig}

\usetheme{confposter} % Use the confposter theme supplied with this template

%\setbeamerfont{block alerted title}{size=\Huge}

\setbeamercolor{block title}{fg=white,bg=black} % Colors of the block titles
\setbeamercolor{block body}{fg=black,bg=white} % Colors of the body of blocks
\setbeamercolor{block alerted title}{fg=black,bg=TopBar} % Colors of the highlighted block titles
\setbeamercolor{block alerted body}{fg=black,bg=white} % Colors of the body of highlighted blocks
% Many more colors are available for use in beamerthemeconfposter.sty

%-----------------------------------------------------------
% Define the column widths and overall poster size
% To set effective sepwid, onecolwid and twocolwid values, first choose how many columns you want and how much separation you want between columns
% In this template, the separation width chosen is 0.01 of the paper width and a 2-column layout
% onecolwid should therefore be (1-(# of columns+1)*sepwid)/# of columns e.g. (1-(2+1)*0.06)/2 = 0.41
% Set twocolwid to be (2*onecolwid)+sepwid = 0.88
% Set threecolwid to be (3*onecolwid)+2*sepwid = N/A

\newlength{\sepwid}
\newlength{\onecolwid}
\newlength{\twocolwid}
\newlength{\threecolwid}
% Ca  peut rester en A0, ca imprime bien sur du A3, A4, ...
\setlength{\paperwidth}{36in} % A0 width: 46.8in
\setlength{\paperheight}{48in} % A0 height: 33.1in
\setlength{\sepwid}{0.06\paperwidth} % Separation width (white space) between columns
\setlength{\onecolwid}{0.41\paperwidth} % Width of one column
\setlength{\twocolwid}{0.88\paperwidth} % Width of two columns
%\setlength{\threecolwid}{0.98\paperwidth} % Width of three columns
\setlength{\topmargin}{-0.3in} % Reduce the top margin size

%-----------------------------------------------------------

\usepackage{graphicx}  % Required for including images

\usepackage{booktabs} % Top and bottom rules for tables

%----------------------------------------------------------------------------------------
%	TITLE SECTION 
%----------------------------------------------------------------------------------------

\title{\includegraphics[width=0.05\linewidth]{Twitter_Bird.png}@reseauloops}% Poster title

\author{LoOPS} % Author(s)

\institute{@reseauloops} % Institution(s)

%----------------------------------------------------------------------------------------

\begin{document}

\addtobeamertemplate{block begin}{}{\vspace*{2ex}} % White space under blocks
\addtobeamertemplate{block end}{}{\vspace*{2ex}} % White space under blocks
\addtobeamertemplate{block alerted begin}{}{\vspace*{2ex}} % White space under highlighted (alert) blocks
\addtobeamertemplate{block alerted end}{}{\vspace*{2ex}} % White space under highlighted (alert) blocks

\setlength{\belowcaptionskip}{3ex} % White space under figures
\setlength\belowdisplayshortskip{3ex} % White space under equations

%% Set default font
\usebeamerfont{block body}

\begin{frame}[t] % The whole poster is enclosed in one beamer frame

\begin{columns}[t] % The whole poster consists of two major columns

\begin{column}{\sepwid}\end{column} % Empty spacer column

\begin{column}{\twocolwid} % Begin a column which is two columns wide (column 2)

%----------------------------------------------------------------------------------------
\vspace{-1.1in}

\begin{alertblock}{
    \vspace{-0.5in}
    \large{Caf\'{e} LoOPS}
    
    \vspace{0.5in}
    \includegraphics[width=0.6\textwidth]{FPGA.jpeg}

    \large{A chacun son processeur maison !}}
    \vspace{-0.5in}
\end{alertblock}

%----------------------------------------------------------------------------------------
\vspace{-1.1in}

\begin{block}{\# date}

{\textcolor{red}{
    %% Date du cafe
    \ding{228} Mardi 12 d\'{e}cembre 2022 de 13h \`{a} 14h.
}}

\end{block}

\begin{block}{\# find}
    %% Lieu du cafe
    \ding{228} Auditorium du B\^{a}timent 100 d'IJCLab

    
\end{block}

\begin{block}{\# ls}
    %% sommaire du cafe
    \ding{228} Vous souhaitez cr\'{e}er votre propre processeur FPGA ?

%% Deplacer cet objet dans le texte pour qu'il soit bien en bas
\begin{wrapfigure}{R}{0.4\textwidth}
    \centering
    \includegraphics[width=0.3\textwidth]{qrcode.png}
    \end{wrapfigure}

    Dans cette présentation, \textbf{Bruno Levy} nous montrera les premières étapes pour démarrer,
    et des pointeurs pour aller plus loin, de la simple LED qui clignotte jusqu'à un
    micro-processeur "FemtoRV" pipeliné capable de faire tourner DOOM !

\end{block}


%----------------------------------------------------------------------------------------

%\begin{columns}[t,totalwidth=\twocolwid] % Split up the two columns wide column again

%\begin{column}{0.6\paperwidth} % The first column within column 2 (column 2.1)

%----------------------------------------------------------------------------------------

%% detail du cafe






\begin{alertblock}{\includegraphics[width=0.065\linewidth]{Twitter_Bird.png}@reseauloops}
\end{alertblock}

%----------------------------------------------------------------------------------------

%\end{column} % End of column 2.1

%\begin{column}{0.32\paperwidth} % The second column within column 2 (column 2.2)

%\begin{figure}
%\vspace{7in}
%\includegraphics[width=1.\linewidth]{tasse.png}

%\end{figure}


%----------------------------------------------------------------------------------------


%\end{column} % End of column 2.2

%\end{columns} % End of the split of column 2

\end{column} % End of the second column


\end{columns} % End of all the columns in the poster

\end{frame} % End of the enclosing frame

\end{document}

